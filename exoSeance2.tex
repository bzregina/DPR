% Options for packages loaded elsewhere
\PassOptionsToPackage{unicode}{hyperref}
\PassOptionsToPackage{hyphens}{url}
%
\documentclass[
]{article}
\usepackage{amsmath,amssymb}
\usepackage{iftex}
\ifPDFTeX
  \usepackage[T1]{fontenc}
  \usepackage[utf8]{inputenc}
  \usepackage{textcomp} % provide euro and other symbols
\else % if luatex or xetex
  \usepackage{unicode-math} % this also loads fontspec
  \defaultfontfeatures{Scale=MatchLowercase}
  \defaultfontfeatures[\rmfamily]{Ligatures=TeX,Scale=1}
\fi
\usepackage{lmodern}
\ifPDFTeX\else
  % xetex/luatex font selection
\fi
% Use upquote if available, for straight quotes in verbatim environments
\IfFileExists{upquote.sty}{\usepackage{upquote}}{}
\IfFileExists{microtype.sty}{% use microtype if available
  \usepackage[]{microtype}
  \UseMicrotypeSet[protrusion]{basicmath} % disable protrusion for tt fonts
}{}
\makeatletter
\@ifundefined{KOMAClassName}{% if non-KOMA class
  \IfFileExists{parskip.sty}{%
    \usepackage{parskip}
  }{% else
    \setlength{\parindent}{0pt}
    \setlength{\parskip}{6pt plus 2pt minus 1pt}}
}{% if KOMA class
  \KOMAoptions{parskip=half}}
\makeatother
\usepackage{xcolor}
\usepackage[margin=1in]{geometry}
\usepackage{graphicx}
\makeatletter
\def\maxwidth{\ifdim\Gin@nat@width>\linewidth\linewidth\else\Gin@nat@width\fi}
\def\maxheight{\ifdim\Gin@nat@height>\textheight\textheight\else\Gin@nat@height\fi}
\makeatother
% Scale images if necessary, so that they will not overflow the page
% margins by default, and it is still possible to overwrite the defaults
% using explicit options in \includegraphics[width, height, ...]{}
\setkeys{Gin}{width=\maxwidth,height=\maxheight,keepaspectratio}
% Set default figure placement to htbp
\makeatletter
\def\fps@figure{htbp}
\makeatother
\setlength{\emergencystretch}{3em} % prevent overfull lines
\providecommand{\tightlist}{%
  \setlength{\itemsep}{0pt}\setlength{\parskip}{0pt}}
\setcounter{secnumdepth}{5}
\newlength{\cslhangindent}
\setlength{\cslhangindent}{1.5em}
\newlength{\csllabelwidth}
\setlength{\csllabelwidth}{3em}
\newlength{\cslentryspacingunit} % times entry-spacing
\setlength{\cslentryspacingunit}{\parskip}
\newenvironment{CSLReferences}[2] % #1 hanging-ident, #2 entry spacing
 {% don't indent paragraphs
  \setlength{\parindent}{0pt}
  % turn on hanging indent if param 1 is 1
  \ifodd #1
  \let\oldpar\par
  \def\par{\hangindent=\cslhangindent\oldpar}
  \fi
  % set entry spacing
  \setlength{\parskip}{#2\cslentryspacingunit}
 }%
 {}
\usepackage{calc}
\newcommand{\CSLBlock}[1]{#1\hfill\break}
\newcommand{\CSLLeftMargin}[1]{\parbox[t]{\csllabelwidth}{#1}}
\newcommand{\CSLRightInline}[1]{\parbox[t]{\linewidth - \csllabelwidth}{#1}\break}
\newcommand{\CSLIndent}[1]{\hspace{\cslhangindent}#1}
\ifLuaTeX
  \usepackage{selnolig}  % disable illegal ligatures
\fi
\IfFileExists{bookmark.sty}{\usepackage{bookmark}}{\usepackage{hyperref}}
\IfFileExists{xurl.sty}{\usepackage{xurl}}{} % add URL line breaks if available
\urlstyle{same}
\hypersetup{
  pdftitle={The Global Competitiveness Report 2019},
  pdfauthor={World Economic Forum},
  hidelinks,
  pdfcreator={LaTeX via pandoc}}

\title{The Global Competitiveness Report 2019}
\author{World Economic Forum}
\date{2021-01-11}

\begin{document}
\maketitle

{
\setcounter{tocdepth}{2}
\tableofcontents
}
\hypertarget{introduction}{%
\section{Introduction}\label{introduction}}

In the long run, a country's economic fortunes are the result of
proactive choices.
\href{https://www3.weforum.org/docs/WEF_TheGlobalCompetitivenessReport2019.pdf}{The
Global Competitiveness Index 4.0} provides stakeholders with a detailed
map of the factors and attributes that drive productivity, growth and
human development. By systematically measuring these intertwined and
complex factors across countries and over time, the GCI offers direction
for policy intervention.

\hypertarget{global-findings-and-implications}{%
\subsection{Global Findings and
Implications}\label{global-findings-and-implications}}

\hypertarget{enhancing-competitiveness-is-still-key-for-improving-living-standards}{%
\subsubsection{Enhancing competitiveness is still key for improving
living
standards}\label{enhancing-competitiveness-is-still-key-for-improving-living-standards}}

Sustained economic growth remains a critical pathway out of poverty and
a core driver of human development. There is overwhelming evidence that
growth has been the most effective way to lift people out of poverty and
improve their quality of life. For least-developed countries (LDCs) and
emerging countries, economic growth is critical for expanding education,
health, nutrition and survival across populations.

\hypertarget{the-global-economy-is-ill-prepared-for-a-downturn-after-a-lost-decade-for-productivity-enhancing-measures}{%
\subsubsection{The global economy is ill-prepared for a downturn after a
lost decade for productivity-enhancing
measures}\label{the-global-economy-is-ill-prepared-for-a-downturn-after-a-lost-decade-for-productivity-enhancing-measures}}

As the shadow of the Great Recession looms large, the global economy is
predicted to be heading for a slowdown. Over the past decade, growth in
advanced economies has been anaemic. Many emerging economies---including
Argentina, India, Brazil, Russia and China---are experiencing some
slowdown or stagnation. In least-developed economies, growth remains
well below potential and highly volatile. Although several factors
explain this lacklustre performance, persistent weaknesses in the
drivers of productivity growth, highlighted by the GCI, are among the
principal culprits.

Productivity growth started slowing down well before the financial
crisis. Between 2000 and 2007, total factor productivity (TFP) annual
growth averaged just 1\% in advanced economies and 2.8\% in emerging and
developing economies. TFP then plummeted during the crisis. Between 2011
and 2016, TFP grew by 0.3\% in advanced economies and 1.3\% in emerging
and developing economies. The financial crisis may actually have
contributed to this deceleration through ``productivity hysteresis'' --
the long-lasting delayed effects of investments being undermined by
uncertainty, low aggregate demand and tighter credit conditions.
Furthermore, beyond strengthening financial system regulations, many of
the structural reforms designed to revive productivity that were
promised by policy-makers in the heat of the crisis did not materialize.

The 2019 results of the GCI 4.0 reveal the size of the deficit in global
competitiveness measures. The average GCI score across the 141 economies
studied is 60.7, measured on a scale of 0 to 100, where 100 is the
``frontier'', an ideal---and hypothetical---situation where a country
achieves the perfect score on every component of the index. In other
words, the global competitiveness gap---measured as the distance to the
frontier---stands at almost 40 points (Figure
1).\includegraphics{./chapter2/images/fig1.png}

Figure 1: The state of global competitiveness in 2019

\hypertarget{country-analysis}{%
\section{Country Analysis}\label{country-analysis}}

This section features regional trends and selected country analysis from
the 2019 edition of the Global Competitiveness Index 4.0. Combining the
GCI scores at a regional level reveals significant differences in both
median competitiveness levels across regions as well as dispersion of
performances within regions.

Overall, the results show that East Asia and the Pacific (17 countries)
achieves the highest median score (73.9) among all regions, followed
closely by Europe and North America (70.9, based on 39 countries).
However, within the East Asia and the Pacific region the competitiveness
gap between the best and worst performers is significantly larger (34.7)
than in Europe and North America (28.9). This shows that, while many
countries in East Asia and the Pacific have come a long way to bring
their competitiveness up to a high level, there are a few that need to
progress faster to bridge their gaps. For instance, comparing the lowest
performers in East Asia and the Pacific and Europe and North America,
Lao PDR's score (50.1) remains about 5 points lower than that of Bosnia
and Herzegovina (54.7). The Middle East \& North Africa, Latin America
and the Caribbean, and Sub-Saharan Africa present similar levels of
dispersion in competitiveness performance (Figure
2).\includegraphics{./chapter2/images/fig2.png}

Figure 2: Competitiveness gap within regions

\hypertarget{selected-country-commentaries}{%
\subsection{Selected country
commentaries}\label{selected-country-commentaries}}

The following section provides an overview of selected economies;

\begin{itemize}
\tightlist
\item
  \textbf{Canada} is \emph{14th} globally, losing two places and 0.3
  points since the 2018 assessment.\\
\item
  \textbf{France} is up two notches over 2018 and now ranks
  \emph{15th}.\\
\item
  \textbf{Japan} ranks \emph{sixth} overall, down one notch over 2018.
\end{itemize}

\hypertarget{conclusion}{%
\section{Conclusion}\label{conclusion}}

The Global Competitiveness Index identifies and assesses the factors
that underpin the process of economic growth and human development. It
highlights the necessity of addressing the spillover effects and
externalities, positive and negative, intended or unintended, of a
policy or strategy beyond the direct objective it pursues. The GCI
encourages the application of systems thinking, an approach that leaders
must adopt in order to apprehend and address today's complex global
challenges. By conceiving of the economy as one of many interacting and
interdependent parts that belong to a vast system, policy-makers have an
opportunity to develop holistic solutions and strategies.(Schwab 2019)

\textbf{References}

\hypertarget{refs}{}
\begin{CSLReferences}{1}{0}
\leavevmode\vadjust pre{\hypertarget{ref-schwab_global_2019}{}}%
Schwab, Klaus. 2019. {``The {Global} {Competitiveness} {Report} 2019,''}
666.

\end{CSLReferences}

\end{document}
